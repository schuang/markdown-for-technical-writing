\hypertarget{markdown-example}{%
\section{Markdown example}\label{markdown-example}}

\hypertarget{lists}{%
\subsection{Lists}\label{lists}}

It's very easy to make some words \textbf{bold} and other words
\emph{italic} with Markdown. You can even \href{http://google.com}{link
to Google!}

\begin{itemize}
\tightlist
\item
  The first important point

  \begin{itemize}
  \tightlist
  \item
    sub-item 1
  \item
    sub-item 2
  \end{itemize}
\item
  The second important point
\end{itemize}

Alternatively,

\begin{itemize}
\tightlist
\item
  The first important point
\item
  The second important point
\end{itemize}

Numbered lists are also supported:

\begin{enumerate}
\def\labelenumi{\arabic{enumi}.}
\tightlist
\item
  First point
\item
  Second point
\end{enumerate}

\hypertarget{links}{%
\subsection{Links}\label{links}}

URL is automatically recognized:

\href{http://github.com}{http://github.com}

Use this syntax to control the displayed text:

\href{http://github.com}{GitHub}

\hypertarget{code}{%
\subsection{Code}\label{code}}

Python code block with syntax highlighting:

\begin{Shaded}
\begin{Highlighting}[]
\KeywordTok{def}\NormalTok{ foo():}
    \BuiltInTok{sum} \OperatorTok{=} \DecValTok{0}
    \ControlFlowTok{for}\NormalTok{ i }\KeywordTok{in}\NormalTok{ [}\DecValTok{1}\NormalTok{,}\DecValTok{2}\NormalTok{,}\DecValTok{3}\NormalTok{,}\DecValTok{4}\NormalTok{]:}
        \BuiltInTok{sum} \OperatorTok{+=}\NormalTok{ i}
    \ControlFlowTok{return} \BuiltInTok{sum}
\end{Highlighting}
\end{Shaded}

\begin{Shaded}
\begin{Highlighting}[]
\DataTypeTok{int}\NormalTok{ main()}
\NormalTok{\{}
    \DataTypeTok{const} \DataTypeTok{int}\NormalTok{ N = }\DecValTok{10}\NormalTok{;}
    \DataTypeTok{int}\NormalTok{ sum = }\DecValTok{0}\NormalTok{;}
    \ControlFlowTok{for}\NormalTok{ (}\DataTypeTok{int}\NormalTok{ i=}\DecValTok{0}\NormalTok{; i\textless{}N; i++) \{}
\NormalTok{        sum += i;}
\NormalTok{    \}}
\NormalTok{    printf(}\StringTok{"sum = \%d}\SpecialCharTok{\textbackslash{}n}\StringTok{"}\NormalTok{, sum);}
    \ControlFlowTok{return} \DecValTok{0}\NormalTok{;}
\NormalTok{\}}
\end{Highlighting}
\end{Shaded}

\hypertarget{table}{%
\subsection{Table}\label{table}}

A table can be easily created:

\begin{longtable}[]{@{}ll@{}}
\toprule
Header 1 & Header 2\tabularnewline
\midrule
\endhead
Content from cell 1 & Content from cell 2\tabularnewline
Content in the first column & Content in the second
column\tabularnewline
\bottomrule
\end{longtable}

Another table style:

\begin{longtable}[]{@{}ll@{}}
\toprule
Header 1 & Header 2\tabularnewline
\midrule
\endhead
Content from cell 1 & Content from cell 2\tabularnewline
Content in the first column & Content in the second
column\tabularnewline
\bottomrule
\end{longtable}
